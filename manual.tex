% Author: Radoslav Grenčík xgrenc00@stud.fit.vutbr.cz


\documentclass[a4paper, 11pt]{article}


\usepackage[czech]{babel}
\usepackage[utf8]{inputenc}
\usepackage[left=2cm, top=3cm, text={17cm, 24cm}]{geometry}
\usepackage{times}
\usepackage{graphicx}
\usepackage[hyphens]{url}
\usepackage[unicode, colorlinks, hypertexnames=false, citecolor=blue]{hyperref}
\usepackage[czech, boxed]{algorithm2e}
\usepackage[perpage]{footmisc}
\usepackage{hyperref}
\usepackage{multirow}
\usepackage[normalem]{ulem}
\useunder{\uline}{\ul}{}


\begin{document}
%##########################################################################%
% TITLE PAGE
%##########################################################################%
\begin{titlepage}
	\begin{center}
		\includegraphics[width=0.77 \linewidth]{FIT_logo.pdf}

		\vspace{\stretch{0.382}}

		\Huge{Simulačná štúdia} \\
		\LARGE{\textbf{Varianta 9: Plasty}} \\

		\vspace{\stretch{0.618}}
	\end{center}

	\begin{minipage}{0.5 \textwidth}
		\Large
		\today
	\end{minipage}
	\hfill
	\begin{minipage}[r]{0.5 \textwidth}
		\Large
		\begin{tabular}{ll}
			Radoslav Grenčík & (xgrenc00) \\
			Róbert Hubinák   & (xhubin03)
		\end{tabular}
	\end{minipage}
\end{titlepage}



%##########################################################################%
% TABLE OF CONTENTS
%##########################################################################%
\clearpage
\thispagestyle{empty}
\tableofcontents



%##########################################################################%
%##########################################################################%
\clearpage
\pagenumbering{arabic}
\setcounter{page}{1}

\section{Úvod}

V tejto práci sa rozoberá problém plastov na našej planéte. Cieľom práce je
vytvoriť model, ktorý popisuje kritickú situáciu s prebytkom plastového odpadu
na našej planéte. V práci sa rozoberá hlavne problém s jednoúčelovými a jednorázovými
plastovými výrobkami ako sú rôzne obaly poprípade iné jednorázové výrobky. Tieto
výrobky tvoria najväčšiu časť plastového odpadu. V práci sa vyskytujú rôzne
experimenty, ktorých zmyslom je demonštrovať, čo sa stane ak okamžite neznížime
produkciu plastového odpadu, ako na množstvo plastového odpadu vplýva recyklácia
a iné faktory.

\subsection{Autori, zdroje}

Projekt vypracovali študenti VUT FIT v Brne Radoslav Grenčík a Róbert Hubinák.

K vypracovaniu projektu boli využité poznatky a študijné texty z predmetu
Modelování a simulace \cite{IMS_slides}, ktorý sa vyučuje na VUT FIT v Brne. Ako zdroj údajov
slúžili rôzne štúdie a články na internete a takisto vlastné meranie.

\subsection{Overovanie validity modelu}

Validita modelu bola overovaná experimentovaním a porovnávaním výsledkov s
reálnymi nameranými dátami, ktoré boli čerpané z overených zdrojov.

%##########################################################################%
%##########################################################################%
\section{Rozbor témy a použitých metód/technológií}

Systém modeluje životný cyklus plastu - od jeho vzniku až po rozklad.
Podľa článku na portále \textbf{Euractiv} \cite{plastic_Europe} celosvetová produkcia
plastu stúpa v roku 2018 bolo vyrobených 359 miliónov ton plastu, čo je 3,2\%
nárast oproti roku 2017.

Vyprodukovaný plast môže byť stále použitý, môže sa z neho stať odpad, môže byť
spálený alebo zrecyklovaný. Podľa článkov na portáloch \textbf{Our World in Data} \cite{plastic_pollution_stats}
a \textbf{ScienceAdvances} \cite{plastic_sciencemag} je približne 30\% plastu stále použitých, približne 56\%
je odpad, približne 8\% je spálených a len približne 6\% je zrecyklovaných.
Ďalej je v týchto článkoch spomenutý fakt, že približne 20\% zo zrecyklovaného
odpadu sa znovu použije, takisto približne 20\% sa spáli a až 60\% zrecyklovaného
odpadu ide na skládky.

Podľa grafov z portálu \textbf{European Parliamentary Research Service Blog} \cite{plastic_graph}
je väčšina plastového odpadu tvorená hlavne plastovými obalmi a druhé miesto
tvoria rôzne plastové výrobky nespadajúce do katégorií: elektronika,
automobilový priemysel ani stavebníctvo. Model sa preto zameriava práve na spomínaný
druh plastového odpadu. Podľa článku na portále \textbf{EcoWatch} \cite{beach_cleanup}
je práve top 10 nájdených vecí pri medzinárodnom čistení pláží hnutím Ocean Conservancy
v roku 2018 plastový odpad a to hlavne cigaretové ohorky a rôzne plastové obaly
alebo iné jednorázové produkty z plastu.

\subsection{Použité postupy}
Pre vytvorenie simulačného modelu sme využili programovací jazyk C++ a klinžnicu SIMLIB \cite{SIMLIB}. Tieto technológie sú vhodné na riešenie nášho problému. Ďalej boli použité postupy popísané v prednáškach k predmetu Modelování a simulace \cite{IMS_slides} na VUT FIT v Brne pre vytvorenie Petriho siete a programovanie v SIMLIBE \cite{SIMLIB}.

\subsection{Popis pôvodu použitých metód a technológii}
Boli použité štandardné knižnice jazyka C++ a knižnica SIMLIB \cite{SIMLIB} pre implementovanie Petriho siete v simulačnom modeli. Autormi knižnice SIMLIB \cite{SIMLIB} sú Petr Peringer, David Leska a David Martinek. Ako nástroj pre preklad bol použitý GNU Make \cite{make}.

%##########################################################################%
%##########################################################################%
\section{Koncepcia metódy, prístupu, modelu}
\label{model:uvod}

Údaj o celosvetovej produkcii plastu bol zjednodušený a v simulačnom modeli sa
generuje každý deň 1 milión ton plastu čo je vo výsledku 365 miliónov ton
plastu ročne. V simulačnom modeli sa dá nastaviť ročný prírastok v produkcii
plastu. Priestupné roky zanedbávame pretože pri takomto množstve je tento
údaj zanedbateľný. Simulačný model si sám počíta čas, za ktorý sa generuje 1
milión ton plastu na základe ročnej produkcie plastu.

V článku na portále \textbf{EcoWatch} \cite{beach_cleanup} sú spomenuté množstvá jednotlivých
vyzbieraných vecí pri medzinárodnom čistení pláží v roku 2018. Na základe týchto množstiev a vlastného merania - približná hmotnosť
predmetov bola získaná vážením rôznych zástupcov určitého druhu a spriemerovaním - bola
vypočítaná celková hmotnosť nájdených predmetov v jednotlivých kategóriách.
Nasledovne boli predmety zoskupené do kategórií podľa doby rozkladu.
Údaje o dobách rozkladu boli problematickým údajom, pretože sa na rôznych stránkach
vyskytujú rôzne údaje. Údaje získané z nasledovných stránok nie sú úplne presné,
avšak pre vytvorenie si predstavy o probléme s plastovým odpadom sú dostačujúce.
Údaje boli získané z nasledovných stránok \cite{decomposition1}, \cite{decomposition2}, \cite{decomposition3}, \cite{decomposition4}, \cite{decomposition5}.
Výsledky sú v tabuľke \ref{tab:1}.

\begin{table}[ht]
	\centering
	\begin{tabular}{llllll}
		\textbf{}                          & \textbf{MNOŽSTVO}                  & \textbf{\begin{tabular}[c]{@{}l@{}}KUSOVÁ\\ HMOTNOSŤ\end{tabular}} & \textbf{\begin{tabular}[c]{@{}l@{}}CELKOVÁ\\ HMOTNOSŤ\end{tabular}} & \textbf{\begin{tabular}[c]{@{}l@{}}DOBA\\ ROZKLADU\end{tabular}} & \textbf{KATEGÓRIA} \\ \hline
		\textbf{\begin{tabular}[c]{@{}l@{}}cigaretový ohorok/\\ drobný odpad\end{tabular}} & 2412151                            & 1,4 g                              & 3377 kg                            & 5-10 rokov                         & A                  \\ \hline
		\textbf{slamka}                    & 643562                             & 0,42 g                             & 270 kg                             & 200 rokov                          & B                  \\ \hline
		\textbf{PET fľaša}                 & 1569135                            & 30 g                               & 47074 kg                           & 450 rokov                          & \multirow{4}{*}{C} \\
		\textbf{PET vrchnák}               & 1091107                            & 2 g                                & 2182 kg                            & 450 rokov                          &                    \\
		\textbf{plastový vrchnák}          & 624878                             & 3 g                                & 1874 kg                            & 450 rokov                          &                    \\
		\textbf{\begin{tabular}[c]{@{}l@{}}"take away"\\ box z plastu\end{tabular}} & 632874                             & 4,5 g                              & 2848 kg                            & 450 rokov                          &                    \\ \hline
		\textbf{igelitová taška}           & 757523                             & 5,5 g                              & 4166 kg                            & 20 rokov                           & \multirow{3}{*}{D} \\
		\textbf{plastové vrece}            & 746211                             & 5,5 g                              & 4104 kg                            & 20 rokov                           &                    \\
		\textbf{fólia/drobný obal}         & 1739743                            & 2 g                                & 3479 kg                            & 20 rokov                           &                    \\ \hline
		\textbf{\begin{tabular}[c]{@{}l@{}}"take away"\\ box z peny\end{tabular}} & 580570                             & 4,5 g                              & 2612 kg                            & 50-80 rokov                        & E                  \\
		\multicolumn{3}{r}{{\ul SPOLU:}}   & \multicolumn{3}{l}{{\ul 71986 kg}}
	\end{tabular}
	\caption{Tabuľka top 10 nájdených predmetov a výsledky meraní}
	\label{tab:1}
\end{table}

Nakoniec bola vypočítaná percentuálna zastúpenosť jednotlivých kategórií v
celkovej hmotnosti vyzbieraného odpadu. Kategória A má zastúpenie 5\%, kategória B 0,4\%,
kategória C 75\%, kategória D 16\% a kategória E 3,6\%. Tieto údaje boli použité pri tvorbe
Petriho siete \ref{appendix:petri_net}.

\subsection{Popis konceptuálneho modelu}

Na vstupe modelu - príloha \ref{appendix:petri_net} sa nachádza časovaný prechod
s dĺžkou prechodu 1 deň. Za túto časovú jednotku sa na vstupe vygeneruje 1 milión
ton plastu. Plast potom môže prejsť do 4 stavov:
\begin{itemize}
	\item pravdepodobnosť 35\% - Plast sa znovu použije a posiela sa na vstup systému.
	\item pravdepodobnosť 56\% - Plast sa stáva odpadom a môže ďalej prejsť do 5 stavov.
	\item pravdepodobnosť 8\% - Plast sa spáli a stáva sa rozloženým - opúšťa systém.
	\item pravdepodobnosť 6\% - Plast sa pošle na recykláciu a môže ďalej prejsť do 3 stavov.
\end{itemize}
Pokiaľ sa plast stal odpadom prejde do jedného z nasledujúcich stavov:
\begin{itemize}
	\item pravdepodobnosť 5\% - Stáva sa cigaretovým ohorkom poprípade iným drobným odpadom. Odpad sa stáva rozloženým za 5-10 rokov - opúsťa systém.
	\item pravdepodobnosť 0,4\% - Stáva sa slamkou. Odpad sa stáva rozloženým za exponenciálne 200 rokov - opúsťa systém.
	\item pravdepodobnosť 75\% - Stáva sa o PET fľašou/vrchnákom poprípade plastovým obalom/vrchnákom. Odpad sa stáva rozloženým za exponenciálne 450 rokov - opúsťa systém.
	\item pravdepodobnosť 16\% - Stáva sa taškou poripáde plastovým vreckom či fóliou. Odpad sa stáva rozloženým za exponenciálne 20 rokov - opúsťa systém.
	\item pravdepodobnosť 3,6\% - Stáva sa penovým "take away" obalom na jedlo. Odpad sa stáva rozloženým za exponenciálne 50 rokov - opúšťa systém.
\end{itemize}
Pokiaľ sa plast pošle na recykláciu prejde do jedného z nasledujúcich stavov::
\begin{itemize}
	\item pravdepodobnosť 20\% - Plast sa spáli a stáva sa rozloženým - opúšťa systém.
	\item pravdepodobnosť 20\% - Plast sa znovu použije a posiela sa na vstup systému.
	\item pravdepodobnosť 60\% - Plast sa stáva odpadom a môže ďalej prejsť do 5 stavov.
\end{itemize}

\subsection{Forma konceptuálneho modelu}

Model je vyjadrený formou Petriho siete - príloha \ref{appendix:petri_net}.

%##########################################################################%
%##########################################################################%
\section{Architektúra simulačného modelu/simulátoru}
\label{architecture:uvod}
Hlavnými komponentami implementačnej časti projektu sú triedy \texttt{Production}
a \texttt{Plastic}. Trieda \texttt{Production} dedí od tiedy \texttt{Event} \cite{SIMLIB}
a stará sa o generovanie a aktiváciu procesov ktoré spracovávame. Životný cyklus
týchto procesov je popísaný v triede \texttt{Plastic}. Program takisto obsahuje triedu
\texttt{ArgumentParser}, ktorá sa stará o spracovanie argumentov programu.

\subsection{Mapovanie konceptuálneho modelu do simulačného modelu}
Ako už bolo spomenuté v úvode kapitoly \ref{architecture:uvod} o generovanie procesov
vstupujúcich do systému sa stará trieda \texttt{Production}. Jeden tento proces predstavuje
jeden milión ton plastu. Po vygrenerovaní je proces rozdelený do jednej zo 4 vetiev, ktoré
predstavujú stavy popísané v modeli \ref{appendix:petri_net} (recyklovaný,skládka...). Rozdelenie
je vo forme intervalov, ktoré zodpovedajú percentám v modeli. O náhodnosť rozdelenia sa stará
funkcia \texttt{Random()} \cite{SIMLIB}. Po tom čo prejde proces do tohto stavu, inkrementuje sa celočíselná premenná
ktorá predstavu množstvo plastu v danom stave. Ak prejde proces do stavu recyklácie je následne
opäť náhodne rozdelení do stavov podľa rozdelenia v modeli. Procesy ktoré sa dostali do stavu skládka
sú takisto rozdelené a podľa kategórie, do ktorej spadajú im je nastavené čakanie funkciou \texttt{Wait()} \cite{SIMLIB}.
Ak takýto čakajúci proces stihne skončiť pred skončením simulácie považujeme ho za rozložený.

\subsection{Spustenie simulačného modelu, parametre programu}
Simulačný model je nutné pred spustením preložiť príkazom make alebo make run (tento príkaz po preklade spustí program). Simulačný model je možné spustiť ako bez parametrov, tak s nimi, a to v ľubovoľnom poradí. Ak užívateľ nezadá parametre, je program spustený s prednastavenými parametrami.

\subsubsection{Popis parametrov programu}
\begin{itemize}
	\item \texttt{-y}\quad Počet rokov simulácie [Prednastavená hodnota: 10 rokov]
	\item \texttt{-r}\quad Percento recyklovaných plastov. Maximálna percentuálna hodnota je nastavená na 62\%, pretože recyklácia sa netýka plastu ktorý je znovapoužitý a spálený. [Prednastavená hodnota: 6 (viz \ref{appendix:petri_net})]
	\item \texttt{-s}\quad Percento úspešne zrecyklovaných plastov(znovupoužitých). Maximálna hodnota je nastavená na 80\% pretože 20\% z recyklovaných plasov sa spáli [Prednastavená hodnota: 20 (viz \ref{appendix:petri_net})]
	\item \texttt{-i}\quad Percentuálny ročný nárast produkcie plastov [Prednastavená hodnota: 0]
\end{itemize}


%##########################################################################%
%##########################################################################%
\section{Podstata simulačných experimentov a ich priebeh}
Cieľom experimentov bolo overiť verejne dostupné informácie o problamatike plastového odpadu vo svete a navrhnúť vhodný a zrealizovateľný plán ako zastaviť nadmerné znečistenie našej planéty. Experimenty 2 - sekcia \ref{label:exp2} a 3 - sekcia \ref{label:exp3} a zameriavajú na otázku, či je vhodnejšie zvýšiť percento recyklovaného odpadu alebo kvalitu recyklácie.

\subsection{Experimenty}

\subsubsection{Experiment 1}
\label{label:exp1}
Cieľom prvého experimentu bolo overiť verejne dostupné informácie zo stránky \textbf{National Geographic} \cite{plastic_recyclation_NATGEO}. Na tejto stránke je uvedené že ak sa bude produkcia plastov naďalej zvyšovať tempom akým sa zvyšuje, v roku 2050 bude na zemi okolo 12000 miliónov ton plastového odpadu. Podľa zdroja \cite{RCOU} bolo v roku 2018 recyklovaných 25 \% plastového odpadu, preto sme nastavili paramter -r na 25. Takisto predpokladáme že úspešnosť recyklácie je aspoň 30\%.Nárast produkcie plastov sa odhaduje na 3.6\% takže parameter -i sme nastavili na 3.6. V článku National Geographic \cite{plastic_recyclation_NATGEO} je takisto spomenuté, že z 6300 milionov ton vyprodukovanych ludstvom do teraz, sa 79 percent uložilo na skládky alebo sa povaľuje voĺne v prírode. Preto sme pred spustením každého experimentu nastavili počiatočnú hodnotu znečistenia na 5000Mton.
\texttt{./ims-projekt -y 50 -r 25 -i 3.6 -s 30}.

\begin{table}[ht]
	\centering
	\begin{tabular}{|l|l|l|l|}
		\hline
		\multicolumn{2}{|l|}{\textbf{Total produced}}    & 18842                & 100\%    \\ \hline
		\multicolumn{2}{|l|}{\textbf{Reused}}            & 7088                 & 38\%     \\ \hline
		\multicolumn{2}{|l|}{\textbf{Waste}}             & 9352                 & 49\%     \\ \hline
		                                                 & \textit{Decomposed}  & 1313   & \\ \hline
		\multicolumn{2}{|l|}{\textbf{Incinerated}}       & 2401                 & 13\%     \\ \hline
		\multicolumn{2}{|l|}{\textbf{Recycled}}          & 4745                 & (20\%)   \\ \hline
		                                                 & \textit{Reused}      & 1444   & \\ \hline
		                                                 & \textit{Incinerated} & 932    & \\ \hline
		                                                 & \textit{Wasted}      & 2368   & \\ \hline
		\multicolumn{2}{|l|}{\textbf{Total world waste}} & 13038 milion tons    &          \\ \hline
	\end{tabular}
	\caption{Výsledok experimentu 1}
	\label{tab:2}
\end{table}

Ako možno vidieť v tabuľke \ref{tab:2}, simulátor vrátil číslo 13038 čo sa takmer zhoduje s údajom v spomínanom článku. Týmto experimentom sme zároveň testovali validitu modelu, pretože v prvých verziách nám simulačný model vracal hodnotu o 5000 väčšiu čím sme odhalili jeho vadu.

\pagebreak
\subsubsection{Experiment 2}
\label{label:exp2}
Cieľom druhého experimnetu bolo zistiť, či je pri snahe o zredukovanie plastového
odpadu výhodnejšie sa zamerať na vyššiu mieru recyklácie odpadu alebo skôr
zefektívniť recykláciu odpadu. Budeme simulovať časový úsek 10 rokov.

Spustenie simulácie s maximálnou mierou recyklácie odpadu -r 62\%: \texttt{./ims-projekt -r 62}.

\begin{table}[ht]
	\centering
	\begin{tabular}{|l|l|l|l|}
		\hline
		\multicolumn{2}{|l|}{\textbf{Total produced}}    & 3590                 & 100\%    \\ \hline
		\multicolumn{2}{|l|}{\textbf{Reused}}            & 1539                 & 43\%     \\ \hline
		\multicolumn{2}{|l|}{\textbf{Waste}}             & 1342                 & 37\%     \\ \hline
		                                                 & \textit{Decomposed}  & 82     & \\ \hline
		\multicolumn{2}{|l|}{\textbf{Incinerated}}       & 708                  & 20\%     \\ \hline
		\multicolumn{2}{|l|}{\textbf{Recycled}}          & 2230                 & (62\%)   \\ \hline
		                                                 & \textit{Reused}      & 461    & \\ \hline
		                                                 & \textit{Incinerated} & 426    & \\ \hline
		                                                 & \textit{Wasted}      & 1342   & \\ \hline
		\multicolumn{2}{|l|}{\textbf{Total world waste}} & 6259  milion tons    &          \\ \hline
	\end{tabular}
	\caption{Experiment 2 - maximálna miera recyklácie}
	\label{tab:3}
\end{table}

Spustenie simulácie s mierou recyklácie odpadu -r 20\% a s maximálnou efektivitou recyklácie -s 100: \texttt{./ims-projekt -r 20 -s 100}.

\begin{table}[ht]
	\centering
	\begin{tabular}{|l|l|l|l|}
		\hline
		\multicolumn{2}{|l|}{\textbf{Total produced}}    & 3590                 & 100\%    \\ \hline
		\multicolumn{2}{|l|}{\textbf{Reused}}            & 1650                 & 46\%     \\ \hline
		\multicolumn{2}{|l|}{\textbf{Waste}}             & 1527                 & 42\%     \\ \hline
		                                                 & \textit{Decomposed}  & 89     & \\ \hline
		\multicolumn{2}{|l|}{\textbf{Incinerated}}       & 413                  & 12\%     \\ \hline
		\multicolumn{2}{|l|}{\textbf{Recycled}}          & 705                  & (20\%)   \\ \hline
		                                                 & \textit{Reused}      & 573    & \\ \hline
		                                                 & \textit{Incinerated} & 132    & \\ \hline
		                                                 & \textit{Wasted}      & 0      & \\ \hline
		\multicolumn{2}{|l|}{\textbf{Total world waste}} & 6438  milion tons    &          \\ \hline
	\end{tabular}
	\caption{Experiment 2 - maximálna efektivita recyklácie}
	\label{tab:4}
\end{table}

Ako možno vidieť v tabuľkách \ref{tab:3} a \ref{tab:4}, simulátor vrátil v tabuľke \ref{tab:3} číslo 6259 a v tabuľke \ref{tab:4} číslo 6438.
V prípade časového úseku 10 rokov sa viacej oplatí zamerať sa na maximálnu efektivitu recyklácie odpadu.

\pagebreak
\subsubsection{Experiment 3}
\label{label:exp3}
Cieľom tretieho experimentu bude overiť čo sa stane ak si zopakujeme experiment číslo 2,
ale budeme simulovať časový úsek 100 rokov.

Spustenie simulácie s maximálnou mierou recyklácie odpadu -r 62\% a časovým úsekom -y 100 rokov: \texttt{./ims-projekt -r 62 -y 100}.

\begin{table}[ht]
	\centering
	\begin{tabular}{|l|l|l|l|}
		\hline
		\multicolumn{2}{|l|}{\textbf{Total produced}}    & 35900                & 100\%    \\ \hline
		\multicolumn{2}{|l|}{\textbf{Reused}}            & 15389                & 43\%     \\ \hline
		\multicolumn{2}{|l|}{\textbf{Waste}}             & 13272                & 37\%     \\ \hline
		                                                 & \textit{Decomposed}  & 3592   & \\ \hline
		\multicolumn{2}{|l|}{\textbf{Incinerated}}       & 7238                 & 20\%     \\ \hline
		\multicolumn{2}{|l|}{\textbf{Recycled}}          & 22177                & (62\%)   \\ \hline
		                                                 & \textit{Reused}      & 4522   & \\ \hline
		                                                 & \textit{Incinerated} & 4380   & \\ \hline
		                                                 & \textit{Wasted}      & 13272  & \\ \hline
		\multicolumn{2}{|l|}{\textbf{Total world waste}} & 14679 milion tons    &          \\ \hline
	\end{tabular}
	\caption{Experiment 3 - maximálna miera recyklácie, 100 rokov}
	\label{tab:5}
\end{table}

Spustenie simulácie s mierou recyklácie odpadu -r 20\%, s maximálnou efektivitou recyklácie -s 100 a časovým úsekom -y 100 rokov: \texttt{./ims-projekt -r 20 -s 100 -y 100}.

\begin{table}[ht]
	\centering
	\begin{tabular}{|l|l|l|l|}
		\hline
		\multicolumn{2}{|l|}{\textbf{Total produced}}    & 35900                & 100\%    \\ \hline
		\multicolumn{2}{|l|}{\textbf{Reused}}            & 16370                & 46\%     \\ \hline
		\multicolumn{2}{|l|}{\textbf{Waste}}             & 15212                & 42\%     \\ \hline
		                                                 & \textit{Decomposed}  & 4130   & \\ \hline
		\multicolumn{2}{|l|}{\textbf{Incinerated}}       & 4317                 & 12\%     \\ \hline
		\multicolumn{2}{|l|}{\textbf{Recycled}}          & 7222                 & (20\%)   \\ \hline
		                                                 & \textit{Reused}      & 5760   & \\ \hline
		                                                 & \textit{Incinerated} & 1463   & \\ \hline
		                                                 & \textit{Wasted}      & 0      & \\ \hline
		\multicolumn{2}{|l|}{\textbf{Total world waste}} & 16081 milion tons    &          \\ \hline
	\end{tabular}
	\caption{Experiment 3 - maximálna efektivita recyklácie, 100 rokov}
	\label{tab:6}
\end{table}

Ako možno vidieť v tabuľkách \ref{tab:5} a \ref{tab:6}, simulátor vrátil v tabuľke \ref{tab:5} číslo 14679 a v tabuľke \ref{tab:6} číslo 16081.
V prípade časového úseku 100 rokov sa taktiež viacej oplatí zamerať sa na maximálnu efektivitu recyklácie odpadu a môžme vidieť že čísla, ktoré vrátil simulátor v tomto experimente sa k sebe nepriblížili, naopak sa od seba viacej vzdialil.
Z tohoto môžme vyvodiť záver, že sa do budúcnosti oplatí zamerať na maximálnu mieru recyklácie odpadu.

\pagebreak
\subsubsection{Experiment 4}
\label{label:exp4}
V tomto experimente sme sa zamerali na to, aký veľký rozdiel spraví 3,6 percentný ročný nárast a úbytok v produkcií plastov. Experimentovali sme s 3 prípadmi. Ako bude situácia vyzerať ak bude produkcia každoročne narastať, klesať a stagnovať. Ako testovacie obdobie sme si zvolili 50 rokov a zvyšné parametre sme nastavili podľa experimentu 1 - sekcia \ref{label:exp1}. Menili sme len parameter -i \{ -3.6, 1.0, 3.6 \}.

\begin{table}[ht]
	\centering
	\begin{tabular}{|l|l|l|l|}
		\hline
		\multicolumn{2}{|l|}{\textbf{Total produced}}    & 48479                & 100\%    \\ \hline
		\multicolumn{2}{|l|}{\textbf{Reused}}            & 18167                & 37\%     \\ \hline
		\multicolumn{2}{|l|}{\textbf{Waste}}             & 24014                & 50\%     \\ \hline
		                                                 & \textit{Decomposed}  & 4577   & \\ \hline
		\multicolumn{2}{|l|}{\textbf{Incinerated}}       & 6297                 & 13\%     \\ \hline
		\multicolumn{2}{|l|}{\textbf{Recycled}}          & 12153                & (25\%)   \\ \hline
		                                                 & \textit{Reused}      & 3640   & \\ \hline
		                                                 & \textit{Incinerated} & 2417   & \\ \hline
		                                                 & \textit{Wasted}      & 6095   & \\ \hline
		\multicolumn{2}{|l|}{\textbf{Total world waste}} & 24436 milion tons    &          \\ \hline
	\end{tabular}
	\caption{Výsledok experimentu s rastúcou produkciou}
	\label{tab:7}
\end{table}

\begin{table}[ht]
	\centering
	\begin{tabular}{|l|l|l|l|}
		\hline
		\multicolumn{2}{|l|}{\textbf{Total produced}}    & 8378                 & 100\%    \\ \hline
		\multicolumn{2}{|l|}{\textbf{Reused}}            & 3139                 & 37\%     \\ \hline
		\multicolumn{2}{|l|}{\textbf{Waste}}             & 4150                 & 50\%     \\ \hline
		                                                 & \textit{Decomposed}  & 791    & \\ \hline
		\multicolumn{2}{|l|}{\textbf{Incinerated}}       & 1088                 & 13\%     \\ \hline
		\multicolumn{2}{|l|}{\textbf{Recycled}}          & 2100                 & (25\%)   \\ \hline
		                                                 & \textit{Reused}      & 629    & \\ \hline
		                                                 & \textit{Incinerated} & 417    & \\ \hline
		                                                 & \textit{Wasted}      & 1053   & \\ \hline
		\multicolumn{2}{|l|}{\textbf{Total world waste}} & 8358 milion tons     &          \\ \hline
	\end{tabular}
	\caption{Výsledok experimentu s klesajúcou produkciou}
	\label{tab:8}
\end{table}

\begin{table}[ht]
	\centering
	\begin{tabular}{|l|l|l|l|}
		\hline
		\multicolumn{2}{|l|}{\textbf{Total produced}}    & 17951                & 100\%    \\ \hline
		\multicolumn{2}{|l|}{\textbf{Reused}}            & 6726                 & 37\%     \\ \hline
		\multicolumn{2}{|l|}{\textbf{Waste}}             & 8892                 & 50\%     \\ \hline
		                                                 & \textit{Decomposed}  & 1695   & \\ \hline
		\multicolumn{2}{|l|}{\textbf{Incinerated}}       & 2332                 & 13\%     \\ \hline
		\multicolumn{2}{|l|}{\textbf{Recycled}}          & 4500                 & (25\%)   \\ \hline
		                                                 & \textit{Reused}      & 1348   & \\ \hline
		                                                 & \textit{Incinerated} & 894    & \\ \hline
		                                                 & \textit{Wasted}      & 2257   & \\ \hline
		\multicolumn{2}{|l|}{\textbf{Total world waste}} & 12196 milion tons    &          \\ \hline
	\end{tabular}
	\caption{Výsledok experimentu so stagnujúcou produkciou}
	\label{tab:9}
\end{table}

Ako môžme vidieť, v experimente kde sme ponechali nárast produkcie 3.6\% - tabuľka \ref{tab:7} je celkový odpad 3krát väčší ako v experimente kde sme znižovali produkciu - tabuľka \ref{tab:8} a 2krát väčší ako v experimente so stagnajúcou produkciou - tabuľka \ref{tab:9}. Zisťujeme, že ak by sme udržali produkciu na leveli akom je momentálne ušetrili by sme 12000 miliónov ton plastového odpadu.

%##########################################################################%
%##########################################################################%
\section{Zhrnutie simulačných experimentov a záver}
Z výsledkov experimentu 1 - sekcia \ref{label:exp1} vyplýva, že náš model je valídny a vracia výsledky odpovedajúce realite.
Z experimentu 2 - sekcia \ref{label:exp2} a experimentu 3 - sekcia \ref{label:exp3} vyplýva, že na vyriešenie problému s plastovým odpadom bude dôležité maximalizovať mieru recyklácie
a taktiež z experimentu 4 - sekcia \ref{label:exp4} vyplýva, že bude nutné zastaviť ročné zvyšovanie produkcie plastov.

V rámci projektu vznikol nástroj, ktorý realisticky modeluje kritickú situáciu s prebytkom plastového odpadu
na našej planéte. Tento nástroj bol implementovaný v jazyku C++ s použitím knižnice SIMLIB \cite{SIMLIB}.
Systém je možné spustiť s rôznymi argumentami a tým pádom vykonávať rôzne experimenty, ktorých výsledky
sú vracané vo formáte prehľadných tabuliek.


%##########################################################################%
% CITATIONS
%##########################################################################%
\clearpage
\bibliographystyle{czechiso}
\renewcommand{\refname}{Literatúra}
\bibliography{manual}



%##########################################################################%
% ADDITIONS
%##########################################################################%
\clearpage
\appendix


% \section{Petriho sieť}
% \label{appendix:petri_net}

% \begin{figure}[ht]
% 	\centering
% 	\includegraphics[width=1 \linewidth]{IMSpetri.pdf}

% 	\caption{Petriho sieť}
% \end{figure}
\end{document}
